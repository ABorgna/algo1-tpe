\documentclass[a4paper]{article}

\usepackage[spanish]{babel} % Le indicamos a LaTeX que vamos a escribir en espa�ol.
\usepackage[latin1]{inputenc} % Permite utilizar tildes y e�es normalmente
%\usepackage{framed}
\usepackage{ifthen}
\usepackage{amssymb}
\usepackage{multicol}
\usepackage{graphicx}
\usepackage[absolute]{textpos}
\makeatletter

\@ifclassloaded{beamer}{%
  \newcommand{\tocarEspacios}{%
    \addtolength{\leftskip}{4em}%
    \addtolength{\parindent}{-3em}%
  }%
}
{%
  \usepackage[top=1cm,bottom=2cm,left=1cm,right=1cm]{geometry}%
  \usepackage{color}%
  \newcommand{\tocarEspacios}{%
    \addtolength{\leftskip}{5em}%
    \addtolength{\parindent}{-3em}%
  }%
}

\newcommand{\encabezadoDeProblema}[4]{%
  % Ponemos la palabrita problema en tt
%  \noindent%
  {\normalfont\bfseries\ttfamily problema}%
  % Ponemos el nombre del problema
  \ %
  {\normalfont\ttfamily #2}%
  \ 
  % Ponemos los parametros
  (#3)%
  \ifthenelse{\equal{#4}{}}{}{%
  \ =\ %
  % Ponemos el nombre del resultado
  {\normalfont\ttfamily #1}%
  % Por ultimo, va el tipo del resultado
  \ : #4}
}

\newcommand{\encabezadoDeTipo}[2]{%
  % Ponemos la palabrita tipo en tt
  {\normalfont\bfseries\ttfamily tipo}%
  % Ponemos el nombre del tipo
  \ %
  {\normalfont\ttfamily #2}%
  \ifthenelse{\equal{#1}{}}{}{$\langle$#1$\rangle$}
}

% Primero definiciones de cosas al estilo title, author, date

\def\materia#1{\gdef\@materia{#1}}
\def\@materia{No especifi\'o la materia}
\def\lamateria{\@materia}

\def\cuatrimestre#1{\gdef\@cuatrimestre{#1}}
\def\@cuatrimestre{No especifi\'o el cuatrimestre}
\def\elcuatrimestre{\@cuatrimestre}

\def\anio#1{\gdef\@anio{#1}}
\def\@anio{No especifi\'o el anio}
\def\elanio{\@anio}

\def\fecha#1{\gdef\@fecha{#1}}
\def\@fecha{\today}
\def\lafecha{\@fecha}

\def\nombre#1{\gdef\@nombre{#1}}
\def\@nombre{No especific'o el nombre}
\def\elnombre{\@nombre}

\def\practicas#1{\gdef\@practica{#1}}
\def\@practica{No especifi\'o el n\'umero de pr\'actica}
\def\lapractica{\@practica}


% Esta macro convierte el numero de cuatrimestre a palabras
\newcommand{\cuatrimestreLindo}{
  \ifthenelse{\equal{\elcuatrimestre}{1}}
  {Primer cuatrimestre}
  {\ifthenelse{\equal{\elcuatrimestre}{2}}
  {Segundo cuatrimestre}
  {Verano}}
}


\newcommand{\depto}{{UBA -- Facultad de Ciencias Exactas y Naturales --
      Departamento de Computaci\'on}}

\newcommand{\titulopractica}{
  \centerline{\depto}
  \vspace{1ex}
  \centerline{{\Large\lamateria}}
  \vspace{0.5ex}
  \centerline{\cuatrimestreLindo de \elanio}
  \vspace{2ex}
  \centerline{{\huge Pr\'actica \lapractica -- \elnombre}}
  \vspace{5ex}
  \arreglarincisos
  \newcounter{ejercicio}
  \newenvironment{ejercicio}{\stepcounter{ejercicio}\textbf{Ejercicio
      \theejercicio}%
    \renewcommand\@currentlabel{\theejercicio}%
  }{\vspace{0.2cm}}
}  


\newcommand{\titulotp}{
  \centerline{\depto}
  \vspace{1ex}
  \centerline{{\Large\lamateria}}
  \vspace{0.5ex}
  \centerline{\cuatrimestreLindo de \elanio}
  \vspace{0.5ex}
  \centerline{\lafecha}
  \vspace{2ex}
  \centerline{{\huge\elnombre}}
  \vspace{5ex}
}


%practicas
\newcommand{\practica}[2]{%
    \title{Pr\'actica #1 \\ #2}
    \author{Algoritmos y Estructuras de Datos I}
    \date{Segundo Cuatrimestre 2015}

    \maketitlepractica{#1}{#2}
}

\newcommand \maketitlepractica[2] {%
\begin{center}
\begin{tabular}{r cr}
 \begin{tabular}{c}
{\large\bf\textsf{\ Algoritmos y Estructuras de Datos I\ }}\\ 
Segundo Cuatrimestre 2015\\
\title{\normalsize Gu\'ia Pr\'actica #1 \\ \textbf{#2}}\\
\@title
\end{tabular} &
\begin{tabular}{@{} p{1.6cm} @{}}
\includegraphics[width=1.6cm]{logodpt.jpg}
\end{tabular} &
\begin{tabular}{l @{}}
 \emph{Departamento de Computaci\'on} \\
 \emph{Facultad de Ciencias Exactas y Naturales} \\
 \emph{Universidad de Buenos Aires} \\
\end{tabular} 
\end{tabular}
\end{center}

\bigskip
}


% Simbolos varios

\newcommand{\ent}{\ensuremath{\mathbb{Z}}}
\newcommand{\float}{\ensuremath{\mathbb{R}}}
\newcommand{\bool}{\ensuremath{\mathsf{Bool}}}
\newcommand{\True}{\ensuremath{\mathrm{True}}}
\newcommand{\False}{\ensuremath{\mathrm{False}}}
\newcommand{\Then}{\ensuremath{\rightarrow}}
\newcommand{\Iff}{\ensuremath{\leftrightarrow}}
\newcommand{\implica}{\ensuremath{\longrightarrow}}
\newcommand{\IfThenElse}[3]{\ensuremath{\mathsf{if}\ #1\ \mathsf{then}\ #2\ \mathsf{else}\ #3}}


\newcommand{\rango}[2]{[#1\twodots#2]}
\newcommand{\comp}[2]{[\,#1\,|\,#2\,]}

\newcommand{\rangoac}[2]{(#1\twodots#2]}
\newcommand{\rangoca}[2]{[#1\twodots#2)}
\newcommand{\rangoaa}[2]{(#1\twodots#2)}

%ejercicios
\newtheorem{exercise}{Ejercicio}
\newenvironment{ejercicio}{\begin{exercise}\rm}{\end{exercise} \vspace{0.2cm}}
\newenvironment{items}{\begin{enumerate}[i)]}{\end{enumerate}}
\newenvironment{subitems}{\begin{enumerate}[a)]}{\end{enumerate}}
\newcommand{\sugerencia}[1]{\noindent \textbf{Sugerencia:} #1}

%tipos basicos
\newcommand{\rea}{\ensuremath{\mathsf{Float}}}
\newcommand{\cha}{\ensuremath{\mathsf{Char}}}

\newcommand{\mcd}{\mathrm{mcd}}
\newcommand{\prm}[1]{\ensuremath{\mathsf{prm}(#1)}}
\newcommand{\sgd}[1]{\ensuremath{\mathsf{sgd}(#1)}}

%listas
\newcommand{\TLista}[1]{[#1]}
\newcommand{\lvacia}{\ensuremath{[\ ]}}
\newcommand{\lv}{\ensuremath{[\ ]}}
\newcommand{\longitud}[1]{\left| #1 \right|}
\newcommand{\cons}[1]{\ensuremath{\mathsf{cons}}(#1)}
\newcommand{\indice}[1]{\ensuremath{\mathsf{indice}}(#1)}
\newcommand{\conc}[1]{\ensuremath{\mathsf{conc}}(#1)}
\newcommand{\cab}[1]{\ensuremath{\mathsf{cab}}(#1)}
\newcommand{\cola}[1]{\ensuremath{\mathsf{cola}}(#1)}
\newcommand{\sub}[1]{\ensuremath{\mathsf{sub}}(#1)}
\newcommand{\en}[1]{\ensuremath{\mathsf{en}}(#1)}
\newcommand{\cuenta}[2]{\mathsf{cuenta}\ensuremath{(#1, #2)}}
\newcommand{\suma}[1]{\mathsf{suma}(#1)}
\newcommand{\twodots}{\ensuremath{\mathrm{..}}}
\newcommand{\masmas}{\ensuremath{++}}

% Acumulador
\newcommand{\acum}[1]{\ensuremath{\mathsf{acum}}(#1)}
\newcommand{\acumselec}[3]{\ensuremath{\mathrm{acum}(#1 |  #2, #3)}}

% \selector{variable}{dominio}
\newcommand{\selector}[2]{#1~\ensuremath{\leftarrow}~#2}
\newcommand{\selec}{\ensuremath{\leftarrow}}


\newenvironment{problema}[4][res]{%
  % El parametro 1 (opcional) es el nombre del resultado
  % El parametro 2 es el nombre del problema
  % El parametro 3 son los parametros
  % El parametro 4 es el tipo del resultado
  % Preambulo del ambiente problema
  % Tenemos que definir los comandos requiere, asegura, modifica y aux
  \newcommand{\requiere}[2][]{%
    {\normalfont\bfseries\ttfamily requiere}%
    \ifthenelse{\equal{##1}{}}{}{\ {\normalfont\ttfamily ##1} :}\ %
    \ensuremath{##2}%
    {\normalfont\bfseries\,;\par}%
  }
  \newcommand{\asegura}[2][]{%
    {\normalfont\bfseries\ttfamily asegura}%
    \ifthenelse{\equal{##1}{}}{}{\ {\normalfont\ttfamily ##1} :}\
    \ensuremath{##2}%
    {\normalfont\bfseries\,;\par}%
  }
  \newcommand{\modifica}[1]{%
    {\normalfont\bfseries\ttfamily modifica\ }%
    \ensuremath{##1}%
    {\normalfont\bfseries\,;\par}%
  }
  \renewcommand{\aux}[4]{%
    {\normalfont\bfseries\ttfamily aux\ }%
    {\normalfont\ttfamily ##1}%
    \ifthenelse{\equal{##2}{}}{}{\ (##2)}\ : ##3\, = \ensuremath{##4}%
    {\normalfont\bfseries\,;\par}%
  }
  \newcommand{\res}{#1}
  \vspace{1ex}
  \noindent
  \encabezadoDeProblema{#1}{#2}{#3}{#4}
  % Abrimos la llave
  \{\par%
  \tocarEspacios
}
% Ahora viene el cierre del ambiente problema
{
  % Cerramos la llave
  \noindent\}
  \vspace{1ex}
}


  \newcommand{\aux}[4]{%
    {\normalfont\bfseries\ttfamily aux\ }%
    {\normalfont\ttfamily #1}%
    \ifthenelse{\equal{#2}{}}{}{\ (#2)}\ : #3\, = \ensuremath{#4}%
    {\normalfont\bfseries\,;\par}%
  }


\newcommand{\pre}[1]{\textsf{pre}\ensuremath{(#1)}}

\newcommand{\problemanom}[1]{\textsf{#1}}
\newcommand{\problemail}[3]{\textsf{problema #1}\ensuremath{(#2) = #3}}
\newcommand{\problemailsinres}[2]{\textsf{problema #1}\ensuremath{(#2)}}
\newcommand{\requiereil}[2]{\textsf{requiere #1: }\ensuremath{#2}}
\newcommand{\asegurail}[2]{\textsf{asegura #1: }\ensuremath{#2}}
\newcommand{\modificail}[1]{\textsf{modifica }\ensuremath{#1}}
\newcommand{\auxil}[2]{\textsf{aux }\ensuremath{#1 = #2}}
\newcommand{\auxilc}[4]{\textsf{aux }\ensuremath{#1( #2 ): #3 = #4}}
\newcommand{\auxnom}[1]{\textsf{aux }\ensuremath{#1}}

\newcommand{\comentario}[1]{{/*\ #1\ */}}

\newcommand{\nom}[1]{\ensuremath{\mathsf{#1}}}

% -----------------
% Tipos compuestos
% -----------------

\newcommand{\Pred}[1]{\mathit{#1}}
\newcommand{\TSet}[1]{\textsf{Conjunto}\ensuremath{\langle #1 \rangle}}
\newcommand{\TSetFinito}[1]{\textsf{Conjunto}\ensuremath{\langle #1 \rangle}}
\newcommand{\TRac}{\tiponom{Racional}}
\newcommand{\TVec}{\tiponom{Vector}}
\newcommand{\Func}[1]{\mathrm{#1}}
\newcommand{\cardinal}[1]{\left| #1 \right|}


\newcommand{\sinonimo}[2]{%
  \noindent%
  {\normalfont\bfseries\ttfamily tipo\ }%
  #1\ =\ #2%
  {\normalfont\bfseries\,;\par}
}

\newcommand{\enum}[2]{%
  \noindent%
  {\normalfont\bfseries\ttfamily tipo\ }%
  #1\ =\ #2%
  {\normalfont\bfseries\,;\par}
}

%~ \newenvironment{tipo}[1]{%
    %~ \vspace{0.2cm}
    %~ \textsf{tipo #1}\ensuremath{\{}\\
    %~ \begin{tabular}[l]{p{0.02\textwidth} p{0.02\textwidth} p{0.82 \textwidth}}
%~ }{%
    %~ \end{tabular}
%~ 
    %~ \ensuremath{\}}
    %~ \vspace{0.15cm}
%~ }
%~ 

\newenvironment{tipo}[2][]{%
  % Preambulo del ambiente tipo
  % Tenemos que definir los comandos observador (con requiere) y aux
  \newcommand{\observador}[3]{%
    {\normalfont\bfseries\ttfamily observador\ }%
    {\normalfont\ttfamily ##1}%
    \ifthenelse{\equal{##2}{}}{}{\ (##2)}\ : ##3%
    {\normalfont\bfseries\,;\par}%
  }
  \newcommand{\requiere}[2][]{{%
    \addtolength{\leftskip}{3em}%
    \setlength{\parindent}{-2em}%
    {\normalfont\bfseries\ttfamily requiere}%
    \ifthenelse{\equal{##1}{}}{}{\ {\normalfont\ttfamily ##1} :}\ 
    \ensuremath{##2}%
    {\normalfont\bfseries\,;\par}}
  }
  \newcommand{\explicacion}[1]{{%
    \addtolength{\leftskip}{3em}%
    \setlength{\parindent}{-2em}%
    \par \hspace{2.3em} ##1 %
    {\par}
    }
  }
  \newcommand{\invariante}[2][]{%
    {\normalfont\bfseries\ttfamily invariante}%
    \ifthenelse{\equal{##1}{}}{}{\ {\normalfont\ttfamily ##1} :}\ 
    \ensuremath{##2}%
    {\normalfont\bfseries\,;\par}%
  }
  \renewcommand{\aux}[4]{%
    {\normalfont\bfseries\ttfamily aux\ }%
    {\normalfont\ttfamily ##1}%
    \ifthenelse{\equal{##2}{}}{}{\ (##2)}\ : ##3\, = \ensuremath{##4}%
    {\normalfont\bfseries\,;\par}%
  }
  \vspace{1ex}
  \noindent
  \encabezadoDeTipo{#1}{#2}
  % Abrimos la llave
  \{\par%
  \tocarEspacios
}
% Ahora viene el cierre del ambiente tipo
{
  % Cerramos la llave
  \noindent\}
  \vspace{1ex}
}


%~ \newcommand{\observador}[3]{%
    %~ & \multicolumn{2}{p{0.85\textwidth}}{\textsf{observador #1}\ensuremath{(#2):#3}}\\%
    %~ }
    
%~ \newcommand{\observador}[3]{%
    %~ {\normalfont\bfseries\ttfamily observador\ }%
    %~ {\normalfont\ttfamily ##1}%
    %~ \ifthenelse{\equal{##2}{}}{}{\ (##2)}\ : ##3%
    %~ {\normalfont\bfseries\,;\par}%
%~ }
    

%~ \newcommand{\observadorconreq}[3]{
    %~ & \multicolumn{2}{p{0.85\textwidth}}{\textsf{observador #1}\ensuremath{(#2):#3 \{}}\\
%~ }
%~ \newcommand{\observadorconreqfin}{
    %~ & \multicolumn{2}{p{0.85\textwidth}}{\ensuremath{\}}}\\
%~ }
%~ \newcommand{\obsrequiere}[2][]{& & \textsf{requiere #1: }\ensuremath{#2};\\}
%~ 
%~ \newcommand{\explicacion}[1]{&& #1 \\}
%~ \newcommand{\invariante}[2][]{%
    %~ & \multicolumn{2}{p{0.85\textwidth}}{\textsf{invariante #1: }\ensuremath{#2}}\\%
%~ }
%~ \newcommand{\auxinvariante}[2]{
    %~ & \multicolumn{2}{p{0.85\textwidth}}{\textsf{aux }\ensuremath{#1 = #2}};\\
%~ }
%~ \newcommand{\auxiliar}[4]{
    %~ & \multicolumn{2}{p{0.85\textwidth}}{\textsf{aux }\ensuremath{#1(#2): #3 = #4}};\\
%~ }

\newcommand{\tiponom}[1]{\ensuremath{\mathsf{#1}}\xspace}
\newcommand{\obsnom}[1]{\ensuremath{\mathsf{#1}}}

% -----------------
% Ecuaciones de terminacion en funcional
% -----------------

\newenvironment{ecuaciones}{%
    $$
    \begin{array}{l @{\ /\ (} l @{,\ } l @{)\ =\ } l}
}{%
    \end{array}
    $$
}




\newcommand{\ecuacion}[4]{#1 & #2 & #3 & #4\\}

\newcommand{\concat}{\nom{concat}}

% Listas por comprension. El primer parametro es la expresion y el
% segundo tiene los selectores y las condiciones.
%*\newcommand{\comp}[2]{[\,#1\,|\,#2\,]}























% En las practicas/parciales usamos numeros arabigos para los ejercicios.
% Aca cambiamos los enumerate comunes para que usen letras y numeros
% romanos
\newcommand{\arreglarincisos}{%
  \renewcommand{\theenumi}{\alph{enumi}}
  \renewcommand{\theenumii}{\roman{enumii}}
  \renewcommand{\labelenumi}{\theenumi)}
  \renewcommand{\labelenumii}{\theenumii)}
}





%%%%%%%%%%%%%%%%%%%%%%%%%%%%%% PARCIAL %%%%%%%%%%%%%%%%%%%%%%%%
\let\@xa\expandafter
\newcommand{\tituloparcial}{\centerline{\depto -- \lamateria}
  \centerline{\elnombre -- \lafecha}%
  \setlength{\TPHorizModule}{10mm} % Fija las unidades de textpos
  \setlength{\TPVertModule}{\TPHorizModule} % Fija las unidades de
                                % textpos
  \arreglarincisos
  \newcounter{total}% Este contador va a guardar cuantos incisos hay
                    % en el parcial. Si un ejercicio no tiene incisos,
                    % cuenta como un inciso.
  \newcounter{contgrilla} % Para hacer ciclos
  \newcounter{columnainicial} % Se van a usar para los cline cuando un
  \newcounter{columnafinal}   % ejercicio tenga incisos.
  \newcommand{\primerafila}{}
  \newcommand{\segundafila}{}
  \newcommand{\rayitas}{} % Esto va a guardar los \cline de los
                          % ejercicios con incisos, asi queda mas bonito
  \newcommand{\anchodegrilla}{20} % Es para textpos
  \newcommand{\izquierda}{7} % Estos dos le dicen a textpos donde colocar
  \newcommand{\abajo}{2}     % la grilla
  \newcommand{\anchodecasilla}{0.4cm}
  \setcounter{columnainicial}{1}
  \setcounter{total}{0}
  \newcounter{ejercicio}
  \setcounter{ejercicio}{0}
  \renewenvironment{ejercicio}[1]
  {%
    \stepcounter{ejercicio}\textbf{\noindent Ejercicio \theejercicio. [##1
      puntos]}% Formato
    \renewcommand\@currentlabel{\theejercicio}% Esto es para las
                                % referencias
    \newcommand{\invariante}[2]{%
      {\normalfont\bfseries\ttfamily invariante}%
      \ ####1\hspace{1em}####2%
    }%
    \renewcommand{\problema}[5][result]{
      \encabezadoDeProblema{####1}{####2}{####3}{####4}\hspace{1em}####5}%
  }% Aca se termina el principio del ejercicio
  {% Ahora viene el final
    % Esto suma la cantidad de incisos o 1 si no hubo ninguno
    \ifthenelse{\equal{\value{enumi}}{0}}
    {\addtocounter{total}{1}}
    {\addtocounter{total}{\value{enumi}}}
    \ifthenelse{\equal{\value{ejercicio}}{1}}{}
    {
      \g@addto@macro\primerafila{&} % Si no estoy en el primer ej.
      \g@addto@macro\segundafila{&}
    }
    \ifthenelse{\equal{\value{enumi}}{0}}
    {% No tiene incisos
      \g@addto@macro\primerafila{\multicolumn{1}{|c|}}
      \bgroup% avoid overwriting somebody else's value of \tmp@a
      \protected@edef\tmp@a{\theejercicio}% expand as far as we can
      \@xa\g@addto@macro\@xa\primerafila\@xa{\tmp@a}%
      \egroup% restore old value of \tmp@a, effect of \g@addto.. is
      
      \stepcounter{columnainicial}
    }
    {% Tiene incisos
      % Primero ponemos el encabezado
      \g@addto@macro\primerafila{\multicolumn}% Ahora el numero de items
      \bgroup% avoid overwriting somebody else's value of \tmp@a
      \protected@edef\tmp@a{\arabic{enumi}}% expand as far as we can
      \@xa\g@addto@macro\@xa\primerafila\@xa{\tmp@a}%
      \egroup% restore old value of \tmp@a, effect of \g@addto.. is
      % global 
      % Ahora el formato
      \g@addto@macro\primerafila{{|c|}}%
      % Ahora el numero de ejercicio
      \bgroup% avoid overwriting somebody else's value of \tmp@a
      \protected@edef\tmp@a{\theejercicio}% expand as far as we can
      \@xa\g@addto@macro\@xa\primerafila\@xa{\tmp@a}%
      \egroup% restore old value of \tmp@a, effect of \g@addto.. is
      % global 
      % Ahora armamos la segunda fila
      \g@addto@macro\segundafila{\multicolumn{1}{|c|}{a}}%
      \setcounter{contgrilla}{1}
      \whiledo{\value{contgrilla}<\value{enumi}}
      {%
        \stepcounter{contgrilla}
        \g@addto@macro\segundafila{&\multicolumn{1}{|c|}}
        \bgroup% avoid overwriting somebody else's value of \tmp@a
        \protected@edef\tmp@a{\alph{contgrilla}}% expand as far as we can
        \@xa\g@addto@macro\@xa\segundafila\@xa{\tmp@a}%
        \egroup% restore old value of \tmp@a, effect of \g@addto.. is
        % global 
      }
      % Ahora armo las rayitas
      \setcounter{columnafinal}{\value{columnainicial}}
      \addtocounter{columnafinal}{-1}
      \addtocounter{columnafinal}{\value{enumi}}
      \bgroup% avoid overwriting somebody else's value of \tmp@a
      \protected@edef\tmp@a{\noexpand\cline{%
          \thecolumnainicial-\thecolumnafinal}}%
      \@xa\g@addto@macro\@xa\rayitas\@xa{\tmp@a}%
      \egroup% restore old value of \tmp@a, effect of \g@addto.. is
      \setcounter{columnainicial}{\value{columnafinal}}
      \stepcounter{columnainicial}
    }
    \setcounter{enumi}{0}%
    \vspace{0.2cm}%
  }%
  \newcommand{\tercerafila}{}
  \newcommand{\armartercerafila}{
    \setcounter{contgrilla}{1}
    \whiledo{\value{contgrilla}<\value{total}}
    {\stepcounter{contgrilla}\g@addto@macro\tercerafila{&}}
  }
  \newcommand{\grilla}{%
    \g@addto@macro\primerafila{&\textbf{TOTAL}}
    \g@addto@macro\segundafila{&}
    \g@addto@macro\tercerafila{&}
    \armartercerafila
    \ifthenelse{\equal{\value{total}}{\value{ejercicio}}}
    {% No hubo incisos
      \begin{textblock}{\anchodegrilla}(\izquierda,\abajo)
        \begin{tabular}{|*{\value{total}}{p{\anchodecasilla}|}c|}
          \hline
          \primerafila\\
          \hline
          \tercerafila\\
          \tercerafila\\
          \hline
        \end{tabular}
      \end{textblock}
    }
    {% Hubo incisos
      \begin{textblock}{\anchodegrilla}(\izquierda,\abajo)
        \begin{tabular}{|*{\value{total}}{p{\anchodecasilla}|}c|}
          \hline
          \primerafila\\
          \rayitas
          \segundafila\\
          \hline
          \tercerafila\\
          \tercerafila\\
          \hline
        \end{tabular}
      \end{textblock}
    }
  }%
  \vspace{0.4cm}
  \textbf{Nro. de orden:}
  
  \textbf{LU:}
  
  \textbf{Apellidos:}
  
  \textbf{Nombres:}
  \vspace{0.5cm}
}



% AMBIENTE CONSIGNAS
% Se usa en el TP para ir agregando las cosas que tienen que resolver
% los alumnos.
% Dentro del ambiente hay que usar \item para cada consigna

\newcounter{consigna}
\setcounter{consigna}{0}

\newenvironment{consignas}{%
  \newcommand{\consigna}{\stepcounter{consigna}\textbf{\theconsigna.}}%
  \renewcommand{\ejercicio}[1]{\item ##1 }
  \renewcommand{\problema}[5][result]{\item
    \encabezadoDeProblema{##1}{##2}{##3}{##4}\hspace{1em}##5}%
  \newcommand{\invariante}[2]{\item%
    {\normalfont\bfseries\ttfamily invariante}%
    \ ##1\hspace{1em}##2%
  }
  \renewcommand{\aux}[4]{\item%
    {\normalfont\bfseries\ttfamily aux\ }%
    {\normalfont\ttfamily ##1}%
    \ifthenelse{\equal{##2}{}}{}{\ (##2)}\ : ##3 \hspace{1em}##4%
  }
  % Comienza la lista de consignas
  \begin{list}{\consigna}{%
      \setlength{\itemsep}{0.5em}%
      \setlength{\parsep}{0cm}%
    }
}%
{\end{list}}



% para decidir si usar && o ^
\newcommand{\y}[0]{\ensuremath{\land}}

% macros de correctitud
\newcommand{\semanticComment}[2]{#1 \ensuremath{#2};}
\newcommand{\namedSemanticComment}[3]{#1 #2: \ensuremath{#3};}


\newcommand{\local}[1]{\semanticComment{local}{#1}}

\newcommand{\vale}[1]{\semanticComment{vale}{#1}}
\newcommand{\valeN}[2]{\namedSemanticComment{vale}{#1}{#2}}
\newcommand{\impl}[1]{\semanticComment{implica}{#1}}
\newcommand{\implN}[2]{\namedSemanticComment{implica}{#1}{#2}}
\newcommand{\estado}[1]{\semanticComment{estado}{#1}}

\newcommand{\invarianteCN}[2]{\namedSemanticComment{invariante}{#1}{#2}}
\newcommand{\invarianteC}[1]{\semanticComment{invariante}{#1}}
\newcommand{\varianteCN}[2]{\namedSemanticComment{variante}{#1}{#2}}
\newcommand{\varianteC}[1]{\semanticComment{variante}{#1}}
% Macros especificas para especificar problemas en AyEDI

\newcommand{\comen}[2]{%
\begin{framed}
\noindent \textsf{#1:} #2
\end{framed}
}
% Aca solo vamos a poner el esqueleto del documento, pero no vamos a especificar nada.

\begin{document} % Todo lo que escribamos a partir de aca va a aparecer en el documento.

\section{Tipos} %Defino los renombres de tipos b�sicos.

\sinonimo{Empleado}{String}
\sinonimo{Energia}{\ent}
\sinonimo{Cantidad}{\ent}
\enum{Bebida}{Pesti Cola, Falsa Naranja, Se ve nada, Agua con Gags, Agua sin Gags}
\enum{Hamburguesa}{McGyver, CukiQueFresco (Cuarto de Kilo con Queso Fresco), McPato, Big Macabra}


\section{Combo} %Defino el tipo combo, con su especificaci�n.


\begin{tipo}{Combo}
	\observador{bebida}{c: Combo}{Bebida}
	\observador{sandwich}{c: Combo}{Hamburguesa}
	\observador{dificultad}{c: Combo}{Energia}
	\medskip % Dejo un espacio entre los observadores y el invariante
	\invariante[dificultadHasta100]{energiaEnRango(dificultad(c))}
\end{tipo}
 % Aca va la definici�n del tipo.

\begin{problema}{nuevoC}{b: Bebida, h: Hamburguesa, d: Energia}{Combo}

\end{problema}

\begin{problema}{bebidaC}{c: Combo}{Bebida}
\end{problema}

\begin{problema}{sandwichC}{c: Combo}{Hamburguesa}
\end{problema}

\begin{problema}{dificultadC}{c: Combo}{Energia}
\end{problema}

 % La especificaci�n del tipo combo.

\newpage %Salto de p�gina

\section{Pedido}

\begin{tipo}{Pedido}
	\observador{numero}{p: Pedido}{\ent}
	\observador{atendio}{p: Pedido}{Empleado}
	\observador{combos}{p: Pedido}{[Combo]}
	
	\medskip
	\invariante[numeroPositivo]{numero(p) > 0}
	\invariante[pideAlgo]{|combos(p)| > 0}
\end{tipo}





\begin{problema}[pedido]{nuevoP}{n: \ent, e: Empleado, cs: [Combo]}{Pedido}
    \requiere{n > 0}
    \requiere{|cs| > 0}
    \asegura{numero(\res) == n}
    \asegura{atendio(\res) == e}
    \asegura{combos(\res) == cs}
\end{problema}

\begin{problema}{numeroP}{p: Pedido}{\ent}
	\asegura{res==numero(p)}
\end{problema}

\begin{problema}{atendioP}{p: Pedido}{Empleado}
 \asegura{res==atendio(p)}
\end{problema}

\begin{problema}{combosP}{p: Pedido}{[Combo]}
	\asegura{\res == combos(p)}
\end{problema}

\begin{problema}{agregarComboP}{p: Pedido, c: Combo}{}
    \modifica{p}
    \asegura{numero(p) == numero(pre(p))}
    \asegura{atendio(p) == atendio(pre(p))}
    \asegura{combos(p) == combos(pre(p)) ++ [c]}
\end{problema}

\begin{problema}{anularComboP}{p: Pedido, i:\ent}{}
\requiere{0 \leq i < |combos(p)|}
    \requiere{|combos(p)|>1}
    \modifica{p}
    \asegura{numero(p) == numero(pre(p))}
    \asegura{atendio(p) == atendio(pre(p))}
    \asegura{combos(p) == [combos(pre(p))[j]|j \leftarrow [0 \: ..\: |combos(pre(p))|), \; j\neq i) \; ]}
\end{problema}

\begin{problema}{cambiarBebidaComboP}{p: Pedido, b: Bebida, i:\ent} {}
    \requiere{0 \leq i < |combos(p)|}
    \modifica{p}
    \asegura{numero(p) == numero(pre(p))}
    \asegura{atendio(p) == atendio(pre(p))}
    \asegura{(\forall \; j \; \leftarrow  [0 \: ..\: |combos(pre(p))|), \; j\neq i) \; combos(p)[j]==combos(pre(p))[j]}
    \asegura{sandwich(combos(p)[i]) == sanwich(combos(pre(p))[i])}
    \asegura{dificultad(combos(p)[i]) == dificultad(combos(pre(p))[i])}
    \asegura{bebida(combos(p)[i]) == b}
\end{problema}

\begin{problema}{elMezcladitoP}{p: Pedido}{}
	\requiere{|combos(p)| \leq |bebidasDe(p)|*|sandwichesDe(p)|}
	\modifica{p}
	\asegura{(\forall i,j \selec combos(p), i \neq j)  \neg comboIgual(combos(p)_i, combos(p)_j)}
	\asegura{|combos(p)| == |combos(pre(p))|}
	\asegura{(\forall c \selec combos(p)) bebida(c) \in bebidasDe(pre(p)))}
	\asegura{(\forall c \selec combos(p)) sandwich(c) \in sandwichesDe(pre(p))}
	
	
	\aux{bebidasDe}{p : Pedido}{[Bebida]}{
		sinRep(bebidasCombo(combos(p)))
	}
	\aux{sandwichesDe}{p : Pedido}{[Hamburguesa]}{
		sinRep(sandwichesCombo(combos(p)))
	}
\end{problema}


\newpage

\section{Local}

\begin{tipo}{Local}
	\observador{stockBebidas}{l: Local, b: Bebida}{Cantidad}
	\requiere{b \in bebidasDelLocal(l)}

	\observador{stockSandwiches}{l: Local, h: Hamburguesa}{Cantidad}
	\requiere{h \in sandwichesDelLocal(l)}

	\observador{bebidasDelLocal}{l:Local}{[Bebida]} 
	\observador{sandwichesDelLocal}{l:Local}{[Hamburguesa]} 
	\observador{empleados}{l: Local}{[Empleado]}
	\observador{desempleados}{l: Local}{[Empleado]}
	\observador{energiaEmpleado}{l: Local, e: Empleado}{Energia}
	\requiere{e \in empleados(l)}

	\observador{ventas}{l: Local}{[Pedido]}
	
	\medskip

	\invariante[hayBebidasySonDistintas]{|bebidasDelLocal(l)|>0 \land distintos(bebidasDelLocal(l))}
	\invariante[haySandwichesySonDistintos]{|sandwichesDelLocal(l)|>0 \land distintos(sandwichesDelLocal(l))}
	\invariante[stockBebidasPositivo]{(\forall b \selec bebidasDelLocal(l)) stockBebidas(l,b) \geq 0 }
	\invariante[stockSandwichesPositivo]{(\forall h \selec sandwichesDelLocal(l)) stockSandwiches(l,h) \geq 0 }
	
	\invariante[empleadosDistintos]{distintos(empleados(l)++desempleados(l))}
	\invariante[energiaHasta100]{(\forall e \selec empleados(l)) energiaEnRango(energiaEmpleado(l,e))}
	\invariante[empleadosQAtendieronDelLocal]{(\forall v \selec ventas(l)) atendio(v) \in empleados(l)++desempleados(l)}
        \invariante[ventasCorrelativas]{mismos(numeros(ventas(l)),rangoNumeros(l))}
	\invariante[combosDeLocal]{\ldots}

        \medskip

        \aux{rangoNumeros}{l : Local}{[\ent]}{
            [min(numeros(ventas(l))) \ldots max(numeros(ventas(l)))]
        }
\end{tipo}



\begin{problema}{stockBebidasL}{l: Local, b:Bebida}{Cantidad}
	\requiere{b \in bebidasDelLocal(l)}
	\asegura{\res == stockBebidas(l,b)}
\end{problema}

\begin{problema}{stockSandwichesL}{l: Local, h:Hamburguesa}{Cantidad}
	\requiere{h \in sandwichesDelLocal(l)}
	\asegura{\res == stockSandwiches(l,h)}
\end{problema}

\begin{problema}[bebidas]{bebidasDelLocalL}{l: Local}{[Bebida]}
    \asegura{mismos(\res,bebidasDelLocal(l))}
\end{problema}

\begin{problema}[sandwiches]{sandwichesDelLocalL}{l: Local}{[Hamburguesa]}
	\asegura{mismos(\res,sandwichesDelLocal(l))}
\end{problema}

\begin{problema}{empleadosL}{l: Local}{[Empleado]}
	\asegura{mismos(res,empleados(l))}
\end{problema}

\begin{problema}{desempleadosL}{l: Local}{[Empleado]}
	\asegura{mismos(res,desempleados(l))}
\end{problema}

\begin{problema}[energia]{energiaEmpleadoL}{l: Local, e:Empleado}{Energia}
    \requiere{e \in empleados(l)}
    \asegura{\res == energiaEmpleado(l,e)}
\end{problema}

\begin{problema}{ventasL}{l: Local}{[Pedido]}
	\asegura{mismos(\res,ventas(l))}
\end{problema}

\begin{problema}{unaVentaCadaUno}{l:Local}{\bool}
	\asegura{res==(\forall \; i \; \leftarrow  [|empleados(l)| \: ..\: |limpiar(l)|) \; ) \; limpiar(l)[i]==limpiar(l)[i-|empleados[l]| \; ] }	
\end{problema}

\begin{problema}{venderL}{l: Local, p:Pedido}{}
	\requiere[siguientePedido]{|ventas(l)| \neq 0 \Rightarrow \\
        numero(p) == max(numsVentas(l)) + 1 \lor numero(p) == min(numsVentas(l))-1}
	\requiere[vendeSandwich]{(\forall x \selec combos(p)) x \in sandwichesDelLocal(l)}
	\requiere[vendeBebida]{(\forall x \selec bebidas(p)) x \in bebidasDelLocal(l)}
	\requiere[stockBebidas]{(\forall b \selec bebidasDelLocal(l)) stockBebidas(l,b) \geq cantidadDeBebida(b,p)}
	\requiere[stockBebidas]{(\forall s \selec sandwichesDelLocal(l)) stockSandwiches(l,s) \geq cantidadDeSandwich(s,p)}
	\requiere{atendio(p) \in empleados(l)}
	
	\medskip	
	
	\modifica{l}
	
	\medskip

	\asegura{bebidasDelLocal(l) == bebidasDelLocal(pre(l))}
	\asegura{sandwichesDelLocal(l) == sandwichesDelLocal(pre(l))}
	\medskip	
	
	\asegura{\neg sePuede(pre(l),p) \Rightarrow mismos(sacar(atendio(p),empleados(pre(l))), empleados(l) )}
	\asegura{\neg sePuede(pre(l),p) \Rightarrow mismos(atendio(p):desempleados(pre(l)), desempleados(l))}
	\asegura{\neg sePuede(pre(l),p) \Rightarrow \\
		(\forall e \selec empleados(l)) energiaEmpleado(l,e) == energiaEmpleado(pre(l),e)}
	\asegura{\neg sePuede(pre(l),p) \Rightarrow \\
		(\forall b \selec bebidasDelLocal(l))stockBebidas(l,b) == stockBebidas(pre(l),b)}
	\asegura{\neg sePuede(pre(l),p) \Rightarrow \\
		(\forall s \selec sandwichesDelLocal(l))stockSandwiches(l,s) == stockSandwiches(pre(l),s)}
	\asegura{\neg sePuede(pre(l),p) \Rightarrow ventas(l) == ventas(pre(l))}
	
	\medskip
	
	\asegura{sePuede(pre(l),p) \Rightarrow empleados(l) == empleados(pre(l))}
	\asegura{sePuede(pre(l),p) \Rightarrow desempleados(l) == desempleados(pre(l))}
	\asegura{sePuede(pre(l),p) \Rightarrow \\
		(\forall e \selec empleados(l), e \neq atendio(p)) energiaEmpleado(l,e) == energiaEmpleado(pre(l),e)}
	\asegura{sePuede(pre(l),p) \Rightarrow \\
		energiaEmpleado(l,atendio(p)) == energiaEmpleado(pre(l),atendio(p)) - energiaPedido(p)}
	\asegura{sePuede(pre(l),p) \Rightarrow \\ 
		(\forall b \selec bebidasDelLocal(l))stockBebidas(l,b) == stockBebidas(pre(l),b) - cantBebida(b,p)}
	\asegura{sePuede(pre(l),p) \Rightarrow  \\
		 (\forall s \selec sandwichesDelLocal(l))stockSandwiches(l,s) == stockSandwiches(pre(l),s) - cantSandwich(s,p)}
	\asegura{sePuede(pre(l),p) \Rightarrow mismos(ventas(l),p:ventas(pre(l)))}
	
	\medskip
	
	\aux{energiaPedido}{p : Pedido}{\ent}{
		\sum [dificultad(x) | x \selec combos(p)]
	}
	\aux{sePuede}{l : Local, p : Pedido}{\bool}{
		energiaEmpleado(l,atendio(p)) > energiaPedido(p)
	}
\end{problema} %auxilio, no entra en la pantalla todo y no se me ocurre como achicarlo mas D:

\begin{problema}{candidatosAEmpleadosDelMesL}{l: Local}{[Empleado]}
    \asegura{|empleados(l)| == 0 \Rightarrow \res == []}
    \asegura{|empleados(l)| > 0 \Rightarrow mismos(\res, [e \: | \: e \selec empleados(l), \\
    ventasHechas(l,e) == maxVentas(l), combosVendidos(l,e) == maxCombos(l,maxVentas(l))])}

    \medskip

    \aux{maxVentas}{l : Local}{\ent}{
        max([ventasHechas(l,e) \: | \: e \selec empleados(l)])
    }
    \aux{maxCombos}{l : Local, ventas : \ent}{\ent}{ \\
        max([combosVendidos(l,e) \: | \: e \selec empleados(l), ventasHechas(l,e) == ventas])
    }
\end{problema}

\begin{problema}{sancionL}{l: Local, e:Empleado, n:Energia}{}
	 \requiere{n \geq 0}
	 \requiere{e \in empleados(l)}
	 
	 \medskip
 
 	 \modifica{l}
	 
	 \medskip

	 \asegura{mismos(bebidasDelLocal(l),bebidasDelLocal(pre(l)))}
	 \asegura{mismos(sandwichesDelLocal(l),sandwichesDelLocal(pre(l)))}
	 \asegura{mismos(ventas(l),ventas(pre(l)))}
	 \asegura{(\forall b \selec bebidasDelLocal(l)) stockBebidas(l,b) == stockBebidas(pre(l),b)}
	 \asegura{(\forall h \selec sandwichesDelLocal(l)) stockSandwiches(l,h) == stockSandwiches(pre(l),h)}
	 \asegura[sigueSiendoEmpleado]{energiaEmpleado(pre(l),e)-n \geq 0 \Rightarrow \\
	 (mismos(empleados(l),empleados(pre(l))) \: \land \:  mismos(desempleados(l),desempleados(pre(l))) \\
	 \land \: energiaEmpleado(l,e) == energiaEmpleado(pre(l),e)-n)}
	 \asegura[esDespedido]{energiaEmpleado(pre(l),e)-n < 0 \Rightarrow \\
	 (mismos(empleados(l),sacar(e,empleados(pre(l)))) \: \land \:  mismos(desempleados(l),e:desempleados(pre(l))))}
	  \asegura{(\forall x \selec empleados(l), x \neq e) enegiaEmpleado(l,x) == enegiaEmpleado(pre(l),x)}
\end{problema}

\begin{problema}{elVagonetaL}{l: Local}{Empleado}	
   \asegura{maximodescanso(l,res)=max([maximodescanso(l,empleados(l)[i])|i \leftarrow [0..|empleados(l)|)])}
   
	\medskip   
   
   \aux{maximodescanso}{l : Local, e : Empleado}{int}{max([apar(l,e)[i]-apar(l,e)[i-1]-1 | i\leftarrow [1..|apar(l,e)|)])}
   \aux{apar}{l : Local, e : Empleado}{[int]}{[0]++[i+1 | i \leftarrow [0..|limpiar(l)|), \: limpiar(l)[i]==e]++ [|limpiar(l)|+1]}   
\end{problema}

\begin{problema}{anularPedidoL}{l: Local, n: \ent}{}
	
	\requiere[pedidoValido]{(\exists p \selec ventas(l)) numero(p)==n}
	\requiere[sigueEmpleado]{empleado(l,n) \in empleados(l)}
	
    \medskip

    \modifica{l}
	
	\medskip
   
    \asegura{empleados(l) == empleados(pre(l))}
    \asegura{desempleados(l) == desempleados(pre(l))}
    \asegura{bebidasDelLocal(l) == bebidasDelLocal(pre(l))}
    \asegura{sandwichesDelLocal(l) == sandwichesDelLocal(pre(l))}
    \asegura{(\forall e \selec empleados(l), e \neq empleado(l,n)) energiaEmpleado(l,e) == energiaEmpleado(pre(l))}
    \asegura{anteriorEnergia(pre(l),n) < 100 \Rightarrow energiaEmpleado(l,empleado(l,n)) == anteriorEnergia(pre(l),n) }
    \asegura{anteriorEnergia(pre(l),n) \geq 100 \Rightarrow energiaEmpleado(l,empleado(l,n)) == 100}
    \asegura{(\forall b \selec bebidasDelLocal(l)) \\
    	stockBebidas(l,b) == stockBebidas(pre(l),b) + cantBebida(b,pedido(pre(l),n))}
    \asegura{(\forall s \selec sandwichesDelLocal(l)) \\
    	stockSandwiches(l,s) == stockSandwiches(pre(l),s) + cantSandwich(s,pedido(pre(l),n))}
    \asegura[correlatividadMenores]{\\(\forall x \selec ventas(pre(l)), numero(x)<n) x \in ventas(l)}
    \asegura[correlatividadMayores]{\\(\forall x \selec ventas(pre(l)), numero(x)>n) (\exists y \selec ventas(l)) igualesSNumer(x,y) \land numero(x)==numero(y)+1}
    \asegura{|ventas(l)|==|ventas(pre(l))-1|}
    
    \medskip
    
    \aux{anteriorEnergia}{l : Local, n : \ent}{\ent}{
    	energiaEmpleado(l,empleado(l,n)) + energiaPedido(pedido(l,n))
    }
    \aux{pedido}{l : Local, n : \ent}{Pedido}{
    	cab([p \: | \: p \selec ventas(l), numero(p) == n])
    }
    \aux{empleado}{l : Local, n : \ent}{Empleado}{
    	atendio(pedido(l,n))
    }
\end{problema}

\begin{problema}{agregarComboAlPedidoL}{l: Local, c: Combo, n:\ent}{}
    \requiere[pedidoValido]{(\exists p \selec ventas(l)) numero(p)==n}
    \requiere[hayBebidas]{bebida(c) \in bebidasDelLocal(l) \land stockBebidas(l,bebida(c)) > 0}
    \requiere[haySandwiches]{sandwich(c) \in sandwichesDelLocal(l) \land stockSandwiches(l,sandwich(c)) > 0}
    \requiere[sigueEmpleado]{empleado(l,n) \in empleados(l)}
    \requiere[energiaSuficiente]{energiaEmpleado(l,empleado(l,n)) \ge dificultad(c)}

    \medskip

    \modifica{l}

    \medskip

    \asegura{mismos(bebidasDelLocal(l), bebidasDelLocal(pre(l)))}
    \asegura{mismos(sandwichesDelLocal(l), sandwichesDelLocal(pre(l)))}
    \asegura{mismos(empleados(l), empleados(pre(l))}
    \asegura{mismos(desempleados(l), desempleados(pre(l))}
    \asegura{
        (\forall b \in bebidasDelLocal(l)) \\
        stockBebidas(l,b) == stockBebidas(pre(l),b) - \beta(b == bebida(c))
    }
    \asegura{
        (\forall s \in sandwichesDelLocal(l)) \\
        stockSandwiches(l,s) == stockSandwiches(pre(l),s) - \beta(s == sandwich(c))
    }
    \asegura[conservacionDeLaEnergia]{
        (\forall e \in empleados(l)) \\
        energiaEmpleado(l,e) == energiaEmpleado(pre(l),e) - \beta(e == empleado(pre(l),n)) * dificultad(c)
    }
    \asegura[conservaVentas]{ \\
        |ventas(l)| == |ventas(pre(l))| \land \\
        (\forall p \selec ventas(pre(l)), numero(p) \neq n) \: (\exists q \in ventas(l)) \: p == q
    }
    \asegura[agregaCombo]{ (\exists p \in ventas(l)) \\
        numero(p) == n \land \\
        atendio(p) == atendio(pedido(pre(l),n)) \land \\
        combos(p) == combos(pedido(pre(l),n)) ++ [c]
    }

    \medskip

    \aux{pedido}{l : Local, n : \ent}{Pedido}{
        cab([p \: | \: p \selec ventas(l), numero(p) == n])
    }
    \aux{empleado}{l : Local, n : \ent}{Empleado}{
        atendio(pedido(l,n))
    }
\end{problema}



\newpage

\section{Funciones Auxiliares}

\aux{distintos}{ls:[T]}{\bool}{ 
  (\forall i,j \selec [0..|ls|), i \neq j) ls_i \neq ls_j
}


\aux{energiaEnRango}{e: Energia} {\bool}{
        0 \leq e \leq 100
}

\subsection{Combo}
% los aux del tipo combo

\subsection{Pedido}

% los aux del tipo pedido

\subsection{Local}

% los aux del tipo local






\end{document} %Termin�!

